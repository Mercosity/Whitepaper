\subsection{Gridcoin Power Consumption estimate}

We start to calculate the consumption of Gridcoin closest relative, Bitcoin. We use first the approach explained in [29], where it is assumed that an ASIC miner (a dedicated hash processor) calculates with 1 Watt of power one Gigahash/s. According to [26], bitcoin speed on August 26, 2017 was 6'354'668.57 terahash/s or 6'354'668'570 gigahash/s or 6'354'668'570 W. These are 6.354 GW of power. They are equivalent to the power produced by 6 large nuclear power plants. Over one year, which has 365 * 24 hours (=8760 hours), we get 6.354 GW * 8760h = 55'661 GWh or 55.661 TWh/year.\\ 

By comparison, the Swiss Federal Railways consume about 3 TWh/year, and CERN in Geneva 1TWh/year.

Another approach outlined in [27] for the Bitcoin Energy Consumption Index, they get 16.2 TWh/year as consumption estimate for last year.

\begin{itemize}
	\item Calculate total mining revenues
	\item Estimate what part is spent in electricity
	\item Find out how much miners pay per kWh
	\item Convert costs to consumption
\end{itemize}

The Bitcoin Energy Consumption Index assumes that miners will ultimately spend 60\% of their revenues on operational costs on average. For every 5 cents that were spent on operational costs it is assumed that 1 kilowatt-hour (kWh) was consumed. [27]\\

Price movements can be small or large, but new energy-hungry machines won't all appear overnight. Realistic behaviour is introduced by linking price dynamics to the expected time required for producers to fully respond to a changing situation. [27]

//TODO: add the same for Ethereum\\


//TODO: add estimate for BOINC users\\
//TODO: add estimate for gridcoin users\\

//TODO: Calculate CO2 impact. 