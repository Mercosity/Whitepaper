\subsection{Gridcoin Funded Science}

Although the scientific method is the cornerstone of modern society, it has also some dark sides or at least it can be futher ameliorated. 

The advent of Internet did a lot to increase communication between scientists. It also introduced the problem of plagiarism [TODO]. For the increasing amount of papers, there is not enough and competent people to guarantee an indipendent and competent peer review [TODO]. Sometimes, papers are not available for free but only through expensive subscribtions (remember Aaron Schwartz [TODO]). It is possible to read a paper, but seldomly the source code of software in the paper is made available to the public. So, it is very difficult to test and indipendently what it is written in the paper, unless a considerable amount of work is invested to replicate the software. Datasets are often kept secret to discourage competitors, although most of the time they are put togheter with many from the taxpayer. The significance problem [TODO] points to another problem: it is easy to collect money for mastodontic projects inline with the mainstream of science thinking, but it is very difficult to get even little funding to test an idea which is outside of mainstream. As a personal opinion: huge amounts of money are spent in search of dark matter, while the electrodynamic theories of Universe are disregarded. (TODO: maybe remove this sentence)\\

If gridcoin would introduce a fix amount in the coinbase of each block with empty input and output a special gridcoin address named 'Gridcoin Funding', the network would collect gridcoins to that special address for each block added to the blockchain. People with an idea who would like to get funded, they would first submit their proposal in form of a whitepaper to the gridcoin community plus a gridcoin address to receive fundings plus the amount of gridcoins needed to fullfill the project. If the proposal fits some basic prerequisites, a special Gridcoin poll will be created on the blockchain asking the community to get funds for the project. In this special poll the gridcoin address of the project and the requested sum of gridcoins will be hardcoded. If the community approves the poll, funds will be automatically sent to that gridcoin address.\\

Getting all fundings in the beginning are normally a bad motivator. So there will be a mechanism which will pay out the amount to the gridcoin address split in fixed intervals, for example monthly. There will be a mechanism to issue a second poll to ask to stop of fundings, in case the project is not performing as expected.\\

On September 10, 2017, the market capitalization of Bitcoin was 67'641'887'163 $ composed by a circulating supply of 16'556'575 of bitcoins at a price of 4'085.50 $. If one percent of that bitcoins would have been spent to a similar fund described above, the fund would have about 676 million dollars available for projects.\\

Malaria is a diffuse disease in Africa but pharmaceutical companies are not investing in medecines for it, not because they are bad as in any good conspiracy theory, but simply because they can not afford to pay the research and development bill with the money they would collect from poor people with that disease. Imagine for a moment a pharmaceutical company asking for a 100 million dollars from the gridcoin fund to start research on malaria cures. Although utopic, the scenario is not completely unthinkable, viewed the numbers of bitcoin in the previous paragraph.\\

Or imagine a hypothetical researcher in Electrodynamics using gridcoin funds unifying electromagnetism and gravitation and wiping out some phantastillion tons of dark matter from the Universe.\\

Or imagine Elon Musk funding the settlement of humans on Mars with gridcoins.\\

Having traditionally funded science compete against gridcoin funded science could spark the next scientific revolution since the Age of Enlightenment.\\

\comm{
brod 
[10:01 PM] 
I would just rewrite gridcoin logic on top of bitcoin.
1 reply Today at 10:05 PM View thread


dangermouse77 [10:02 PM] 
my colleague at work told me that it is possible to rewrite steem on ethereum


brod 
[10:02 PM] 
**rewrite**


[10:02] 
it is possible


dangermouse77 [10:02 PM] 
so i think it would be possible to rewrite gridcoin logic on ethereum


[10:03] 
but fortunately this would be very slow


brod 
[10:03 PM] 
I thing it would be even slower than on top of bitcoin


dangermouse77 [10:03 PM] 
dogecoin was standalone and now it is on top of ethereum
1 reply Today at 10:06 PM View thread


cm
[10:05 PM] 
If we were to seriously consider doing something like this, wouldn't it make sense to also consider alternative codebases such as graphene/eos/hyperledger etc?



cm
[10:06 PM] 
Doge runs on top of Eth?! How did I miss this news? lol..


dangermouse77 [10:06 PM] 
cm yesterday i had another crazy idea, i prepared a steemit article but then i let it drop


[10:06] 
i did enough confusion for the moment


dangermouse77 [10:06 PM] 
still can i tell you the idea?
1 reply Today at 10:07 PM View thread


cm 
[10:06 PM] 
Sure the doge on eth wasn't an april fools joke? :stuck_out_tongue:


dangermouse77 [10:07 PM] 
i am not sure. my colleague at work is fan of ethereum and knows it in deep


[10:07] 
for example he explained me how to implement steem on ethereum:


brod 
[10:07 PM] 
You are really dangerous



dangermouse77 [10:08 PM] 
create a smart contract with three methods: listAllArticles(), insertArticle(), readArticle() (edited)


[10:08] 
listAllArticles simply retrieves all articles, yes it is easy


[10:09] 
insertArticle: it has as parameter the article, and the ethereum address for your reward


[10:09] 
sorry i messed up with the third call, the third call is readArticle, of course


[10:10] 
to call readArticle you need to put which article you want to read, and pay an amount for ethereum which depends on the popularity of the author, the amount of reads, the upvotes etc.


brod 
[10:10 PM] 
how do you list all articles? were you get the data from? how you protect against invalid data?
1 reply Today at 10:27 PM View thread


cm 
[10:11 PM] 
Simply wouldn't work, ETH has 30 TPS max where as steem has 10k TPS+ plus it would cost money to operate on top of ETH where as steem is entirely free to post on.


dangermouse77 [10:11 PM] 
the ethereum paid in insertArticle goes to the address which was input parameter in insertArticle


[10:11] 
cm, what does TPS mean?


cm 
[10:11 PM] 
transactions per second


dangermouse77 [10:11 PM] 
ah transactions per second


[10:12] 
wow how much TPS does gridcoin have?


[10:12] 
7 like bitcoin?


cm 
[10:12 PM] 
yeah, if that


dangermouse77 [10:12 PM] 
only 7? :disappointed:


[10:13] 
Visa has 30k TPS


cm 
[10:13 PM] 
bitcoin cash has like 60 TPS given the 8mb blocks now, still tiny though compared to visa


dangermouse77 [10:13 PM] 
anyway 7 TPS is enough for a coin to act as "value storage", but not for everyday needs...


[10:13] 
still i stick with gridcoin, because...


[10:14] 
because..


[10:14] 
(now i have to tell you about this idea)


[10:15] 
probably you also already had it, because in the list of commands of gridcoinresearchd i saw a DAO command


cm 
[10:15 PM] 
it's gone after the mandatory release AFAIK


dangermouse77 [10:15 PM] 
what does the DAO command do exactly?


cm 
[10:15 PM] 
I think the idea was to trade securities on top of gridcoin via gridcoin finance, but those plans fell through


[10:15] 
probably for the best given sec now going after the DAO


dangermouse77 [10:15 PM] 
ah ok then it is not this idea


[10:16] 
but it has similarities to the DAO of ethereum


[10:16] 
the idea comes from this discussion about BOINC not supporting gridcoin


[10:18] 
because i think it is also a mentality problem: succesful researchers were previously students who almost starved and that endured everything for the superior sake of science, so they are not venial and they do not like to connect science with money. still they enjoy high wages and most of the time they need to consume to get new funds


[10:18] 
i read arstechnica and sometimes they point out at problems that science has now:


[10:19] 
for example plagiarism, not enough and competent people that do peer review


[10:19] 
paying money to get papers (remember Aaron Schwartz)


[10:19] 
datasets not free


[10:20] 
you always can read the paper but you never get the source code! so it is very difficult to test what they tell other to come out with modifications


[10:20] 
people thinking different are outsiders, if you are in mainstream


[10:20] 
you get lot of money, if not you starve


[10:20] 
(in my opinion dark matter and strings are crazy ideas, still they get the funds)


[10:21] 
so what about if we would create beside traditionally funded science


[10:21] 
another branch of science which would compete with traditionally funded science


[10:21] 
this branch would be called: gridcoin funded science


cm 
[10:23 PM] 
Like this? https://steemit.com/boinc/@cm-steem/brainstorming-boinc-projects-003#@cm-steem/re-cm-steem-brainstorming-boinc-projects-003-20161207t232355590z (edited)


[10:23] 
probs worth moving this to #boinc_projects


dangermouse77 [10:24 PM] 
if we would raise gridcoin inflation say from 1.5% to 2% and if we would introduce a fix coinbase for each block representing this 0.5% and pay it to a particular gridcoin science funding address which can be only unlocked with a poll mechanism: people can then do formal requests to the gridcoin team with a project, examples of what they did until now, and how much gridcoin they need. the community would then vote on the poll and decide if to free the funds for that project


[10:26] 
wonder how Musk will put people on Mars? gridcoin funded research :slightly_smiling_face:


dangermouse77 [10:27 PM] 
articles have a hash as id which links to a database. in ethereum you implement only the core business logic. all the rest is done by a linked database and say a web frontend.


dangermouse77 [10:29 PM] 
so bitcoin will stay as pillar of cryptoeconomy, ethereum for commercial companies, steem for publishing industry. but give to gridcoin what is for gridcoin: scientific research


[10:29] 
traditional science and gridcoin science will compete together


[10:29] 
traditional science will go lost sometimes after dark matter.


ifoggz
[10:29 PM] 
Well migrated my office completely to laptop. Was quite the task now to disable secureboot and get better drivers for my shit.


dangermouse77 [10:30 PM] 
gridcoin science will fund some researchers who will research also electrodynamic universe


[10:30] 
also in medicine there is a problem: cancer and the disease of the first world get funded, because the companies need to pay the research bill (edited)


[10:32] 
malaria and other diseases get no fundings, because companies can not afford it as the people in the 3 world can not pay for medecines


[10:32] 
and competition will reign


[10:32] 
i mean: i am a fan of traditional science. without it i would live in a cavern, eat raspberry (pi) and try to stone the bear who attacks my family


dangermouse77 [10:39 PM] 
having two different ways of funding science and get them the two branches competing


[10:39] 
man, i think we will go to the moon. i almost can not wait to have an X-Wing in my garage :slightly_smiling_face: (edited)


jringo [10:41 PM] 
eth is incredibly young and mysterious -- if we were to port I'd wait for eos to secure itself then do eos or graphene.  nxt is amazin


dangermouse77 [10:43 PM] 
would be a problem to add a fix coinbase to each block with funds going to a particular address? to implement a poll mechanism (etheruem would call it a smart contract) to free the funds to another address?


jringo [10:43 PM] 
inflation is another conversation


[10:45] 
i am in favor of block rewards over interest with the goal of 0 or less inflation


[10:45] 
i think using a fraction of the block reward to fund projects is a good idea


[10:46] 
lots of potential
}